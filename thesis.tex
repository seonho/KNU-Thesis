%!TEX encoding = UTF-8 Unicode
%!TEX program = xelatex

\documentclass[b5paper,11pt,oneside,openany,article,no-math]{memoir}
\setsecnumdepth{subsection} % enable subsection numbering

\usepackage{knuthesis}
\setmainfont{Times New Roman} % 기본 글꼴 설정

\usepackage{amsmath, amssymb, mathrsfs} % 수식 조판

% 표를 만드는 패키지
\usepackage{tabu,longtable,multirow,multicol,booktabs}

% use graphics package
\usepackage{graphicx}
% set default figure path
\graphicspath{{./fig/}}

% 논문제목, 저자정보 등
\InputIfFileExists{author.tex}{}{\message{LaTeX Warning: Author information should be written in 'author.tex'.}}

% 하이퍼 링크 설정
\usepackage[
pdfencoding=auto,
psdextra,
pdfusetitle,
colorlinks,
linkcolor=black
]{hyperref}
% 출력시에는 아래 줄을 주석 해제하자.
% \usepackage[hidelinks]{hyperref}
\usepackage{bookmark}

% 알고리즘 관련 패키지
\usepackage{algorithm2e, algorithmic}

\usepackage[super]{nth} % nth

\usepackage{cite} % citation

% 챕터 스타일 지정
\chapterstyle{madsen}

\newsubfloat{figure}% Allow subfloats in figure environment

\makeatletter
% make the figure name and number bold:
\renewcommand{\fnum@figure}{\textbf{\figurename~\thefigure}}
% for section
\renewcommand{\sectionrefname}{Section}
\newcommand{\sref}[1]{\sectionrefname~\ref{#1}}
\newcommand{\cref}[1]{\Cref{#1}}
\newcommand{\aref}[1]{\appendixname~\ref{#1}}
\newcommand{\alref}[1]{Algorithm~\ref{#1}}
\makeatother

\renewcommand{\algorithmicrequire}{\textbf{Input:}}
\renewcommand{\algorithmicensure}{\textbf{Output:}}

\newcommand\Algphase[1]{%
\vspace*{-.7\baselineskip}\Statex\hspace*{\dimexpr-\algorithmicindent-2pt\relax}%
\Statex\hspace*{-\algorithmicindent}\textbf{#1}%
\vspace*{-.7\baselineskip}\Statex\hspace*{\dimexpr-\algorithmicindent-2pt\relax}%
}

\newcommand\Algsubphase[1]{%
% \vspace*{-.7\baselineskip}\Statex\hspace*{\dimexpr-\algorithmicindent-2pt\relax}%
\Statex\hspace*{-\algorithmicindent}\textbf{\small #1}%
% \vspace*{-.7\baselineskip}\Statex\hspace*{\dimexpr-\algorithmicindent-2pt\relax}%
}

\begin{document}

\frontmatter

% 표지
\maketitle
% 인준서
\makeapproval

\setcounter{page}{1}

% 목차 생성
\clearpage
\tableofcontents*
% 그림 목차 생성
\clearpage
\listoffigures*
% 표 목차 생성
\clearpage
\listoftables*

% 출력시에는 아래 줄을 주석 처리하자.
\hypersetup{linkcolor=red}

% 본문 시작
\mainmatter

% 본문 줄간격 지정
\setstretch{1.5}

\clearpage
\chapter{Introduction}

This is a demonstration of KNU thesis template~\cite{knuthesis2015}.

\clearpage
\chapter{Related Work}

\clearpage
\chapter{Conclusion}
% \include{tex/ch1}
% \include{tex/ch2}
% \include{tex/ch3}
% \include{tex/ch4}
% \include{tex/ch5}
% \include{tex/ch6}

\clearpage
\appendix
% \include{tex/appa} % appendix

\backmatter
\bibliographystyle{IEEEtran}
\bibliography{thesis}

\clearpage
\addcontentsline{toc}{chapter}{Abstract} % TOC에 목차 추가
\makekorabstract{\InputIfFileExists{tex/abstractkor.tex}{}{tex/abstractkor.tex를 읽어옵니다.}}

\end{document}